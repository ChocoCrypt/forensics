% article example for classicthesis.sty
\documentclass[10pt,a4paper]{article} % KOMA-Script article scrartcl
\usepackage{lipsum}     %lorem ipsum text
\usepackage{titlesec}   %Section settings
\usepackage{titling}    %Title settings
\usepackage[margin=10em]{geometry}  %Adjusting margins
\usepackage{setspace}
\usepackage{listings}
\usepackage{amsmath}    %Display equations options
\usepackage{amssymb}    %More symbols
\usepackage{xcolor}     %Color settings
\usepackage{pagecolor}
\usepackage{mdframed}
\usepackage[spanish]{babel}
\usepackage[utf8]{inputenc}
\usepackage{longtable}
\usepackage{multicol}
\usepackage{graphicx}
\graphicspath{ {./Images/} }
\setlength{\columnsep}{1cm}

% ====| color de la pagina y del fondo |==== %
\pagecolor{white}
\color{black}



\begin{document}
    %========================{TITLE}====================%
    \title{\rmfamily\normalfont\spacedallcaps{ Respuestas clase 2 }}
    \author{\spacedlowsmallcaps{Rodrigo Castillo junto a David Martinez}}
    \date{\today} 
    
    \maketitle
     

    %=======================NOTES GOES HERE===================%
    \section{Primera razón}

        La primera razon por la cuál alguien querría analizar un malware es
        para buscar a sus creadores, pues muchos ataques informáticos son
        catastróficos para las empresas y por esta razón es bueno buscar a sus
        culpables

    \section{Segunda razón}

        Para dimensionar el daño potencial que pueda tener un archivo malicioso
        en un computador : Muchas veces nosotros descargamos contenido del cuál
        no sabemos su procedencia, por lo tanto, es bueno poder analizar que
        acciones está teniendo este contenido en nuestros dispositivos y de
        esta manera poder hacer un balance para saber si este contenido nos
        beneficia o nos perjudica.

    \section{Tercera Razon}

        Para entender el Malware, entender el malware puede prevenirnos de
        futuros ataques informáticos en muchas ocaciones, nos puede enseñar
        como prevenirlo y en un ambiente de seguridad ofensiva , como
        ejecutarlo.
        
        \\ Además de todo lo anterior, entender el malware puede ser facinante
        y puede contribuir con otras diciplinas de la informática, con esto,
        construir herramientas que nos puedan beneifciar en un futuro.

    \section{Diferencia entre Host based signatures y Antivirus signatures}

        Host based signatures detectan cambios inesperados en el codigo
        malicioso y cambios en las firmas de antivirus, se enfocan en las
        acciones que el malware de hacen al sistema. 
        
        \\ Los antivirus lo que hacen es comparar el malware con el malware
        conocido con el malware que quiere verificar. de esta manera, reconoce
        el malware, sin embargo, el malware puede presentarse en diferentes
        codificaciones, por lo que mediante algoritmos de codificación es
        posible bypassear un antivirus.
        
    







    %=======================NOTES ENDS HERE===================%
    
    % bib stuff
    \nocite{*}
    \addtocontents{toc}{\protect\vspace{\beforebibskip}}
    \addcontentsline{toc}{section}{\refname}    
    \bibliographystyle{plain}
    \bibliography{../Bibliography}
\end{document}
