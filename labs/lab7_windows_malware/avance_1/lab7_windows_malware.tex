% article example for classicthesis.sty
\documentclass[10pt,a4paper]{article} % KOMA-Script article scrartcl
\usepackage{import}
\usepackage{xifthen}
\usepackage{pdfpages}
\usepackage{transparent}
\newcommand{\incfig}[1]{%
    \def\svgwidth{\columnwidth}
    \import{./figures/}{#1.pdf_tex}
}
\usepackage{lipsum}     %lorem ipsum text
\usepackage{titlesec}   %Section settings
\usepackage{titling}    %Title settings
\usepackage[margin=10em]{geometry}  %Adjusting margins
\usepackage{setspace}
\usepackage{listings}
\usepackage{amsmath}    %Display equations options
\usepackage{amssymb}    %More symbols
\usepackage{xcolor}     %Color settings
\usepackage{pagecolor}
\usepackage{mdframed}
\usepackage[spanish]{babel}
\usepackage[utf8]{inputenc}
\usepackage{longtable}
\usepackage{multicol}
\usepackage{graphicx}
\graphicspath{ {./Images/} }
\setlength{\columnsep}{1cm}

% ====| color de la pagina y del fondo |==== %
\pagecolor{white}
\color{black}



\begin{document}
    %========================{TITLE}====================%
    \title{{  7th Lab - Windows malware  }}
    \author{{Rodrigo Castillo and Juan Esteban Murcia}}
    \date{\today}

    \maketitle


    %=======================NOTES GOES HERE===================%
    \section{Read the introduction to the Section "Analyzing Malicious Windows
        Programs" and explain why is it important to know the details of Windows OS
        (Windows API, user/kernel mode, execution of code outside a file)?}

        As a computer science student, is important to know details about
        Windows OS because is the most used operating system , as a forensics
        investigator, is important because most of the malware works for
        windows, somethimes, malware will use those libraries and services
        making the aknowledgment of those libraries really important for
        understanding and studyng malware and how it works.

    \section{Read the section "The Windows API" and identify which are the
        Windows API Types}.

        types of windows api:
        \begin{itemize}
            \item {WORD:}
                is a unsigned 16bit value .
            \item {DWORD:}
                is a unsigned 32bit value .
            \item {HANDLES:}
                is a reference to an object , is not documented so it should
                only be managed by windows APIs.
            \item {Long Pointer}
                is a pointer to another type of variable.
            \item {Callback}
                represents a function that is going to be called by a windows
                API.
        \end{itemize}

    \section{Read the section "File System Functions" and explain the
    difference between shared files and files accesible via namespaces}

    shared files are special files which paths looks like  \textbf{$ //nameserver/share  $}
    \textbf{or}   \textit{\textbf{\textbf{\textbf{$ //?/nameserver/share
    $}}}}. the symbol  tells the operating
    system to  not parse the string, in order to access longer
    filenames.

    Files accesible via namespaces:
    namespaces can be understood as the number of a folder, each one store
    different type of objects.
    \\
    lowest namespace is called \textbf{NT}  and it has access to all devices
    and  all namespaces exist inside \textbf{NT} .
























    %=======================NOTES ENDS HERE===================%

    % bib stuff
    \nocite{*}
    \addtocontents{toc}{{}}
    \addcontentsline{toc}{section}{\refname}
    \bibliographystyle{plain}
    \bibliography{../Bibliography}
\end{document}
