% article example for classicthesis.sty
\documentclass[10pt,a4paper]{article} % KOMA-Script article scrartcl
\usepackage{lipsum}     %lorem ipsum text
\usepackage{titlesec}   %Section settings
\usepackage{titling}    %Title settings
\usepackage[margin=10em]{geometry}  %Adjusting margins
\usepackage{setspace}
\usepackage{listings}
\usepackage{amsmath}    %Display equations options
\usepackage{amssymb}    %More symbols
\usepackage{xcolor}     %Color settings
\usepackage{pagecolor}
\usepackage{mdframed}
\usepackage[spanish]{babel}
\usepackage[utf8]{inputenc}
\usepackage{longtable}
\usepackage{multicol}
\usepackage{graphicx}
\graphicspath{ {./Images/} }
\setlength{\columnsep}{1cm}

% ====| color de la pagina y del fondo |==== %
\pagecolor{white}
\color{black}



\begin{document}
    %========================{TITLE}====================%
    \title{\rmfamily\normalfont\spacedallcaps{ Solucion Laboratorio 1  }}
    \author{\spacedlowsmallcaps{Rodrigo Castillo}}
    \date{\today} 
    
    \maketitle
     

    %=======================NOTES GOES HERE===================%
        Taks : 
        \\ 
        \\ 
        Lab 1 - Cybersecurity Concepts
        \\
        Write one essay of minimum 1 page ("Letter size", Font 11, Single line) in
        academic English where you address the following assignment:
        \\
        An e-shop company with a national presence hosts its web servers on its own
        data center. The e-shop has been working fine until now, however, the
        growth in sales, due partially to the Free-taxes days, have motivated the
        company to migrate to the cloud considering the possible benefits: i)
        scalability (to support the most demanding days), ii) flexibility (to make
        fast deployments), iii) location (to reach new costumer) and iv)costs (Only
        pay for what is consumed without Cost-of-Ownership). You, a cybersecurity
        consultant, must define the Security Strategy that the company should
        implement to migrate to the cloud in a safe and functional way.
        \\
        In the essay, you must include concepts you have learned in the 4 challenges
        studied in AWS Educate along this week: i) components of cloud computing, ii)
        multifactor authentication (MFA), iii) least privilege, iv) personal data
        privacy, v) encryption, vi) vpn, vii) https vs http, viii) digital certificate
        to validate identity, ix) password manager, x) integrity, xi) confidentiality,
        xii) availability, etc.
        \\
        Use the free service of Grammarly to review and adjust your academic
        English before submitting the essay.
    \section{Basic Stuff}
        At first, the E-Shop must make physical backups of their information,
        so if they make any mistake making the transition they are not going to
        loose their Data. 
    \section{Costs}
        Second they have to browse for good alternatives to
        cloud services. As they are an E-Shop, thei'r data must be
        confidential, so they must include how much confidence are they able to
        put in thei'r cloud service client on the criteria of picking a cloud
        service. 
     \section{Scalability and location}
        They also most include how scalability works in their cloud
        service because there are cloud services that are super cheap, but the
        scalability is expensive, and they also have to include the location of
        the cloud server service, because, for example, if they are in latin
        america and they offer a service in latin america, getting a service in
        other contintents can cause an slow service and thats not going to be
        worse for them. Also they have to check the state of the server in the
        cloud service. I think a good way for checking this is asking for free
        trials and checking them before paying for a cloud service. 

     \section{Basics about least priviledge concept}
        \\Once they know which service are they going to pick, they will have to
        check that their service have minimum standarts of security. Check
        that every person who have contact with the service have the minimum
        priviledges so they can use the service but they cant abuse of it,
        there are many standarts of minimum priviledges principes so the
        company can pick it accordint to their necesities. 

    \section{Cypher }
        Then
        they have to check that the comunication between the clients and the
        server is correctly cypher so there is not going to be intruders
        intercepting the comunication between the users and the service. 
        
        \\they will have to verify that their databases are properly encrypted
        , by correctly hashing the passwords and using correcs and
        non-vulnerable hashing algorithms for this. i will explain this later
        in the section of symmetric encryption

        \subsection{asymmetric encryption}

            \\ Website must have proper encription algorithms and enought bits of
            RSA encription so they will be able to be secure at the eyes of the
            public. Also they have validate certificates so customers can get sure
            that the page is not being spoofed.

            \\ Sometimes, vulnerabilities occur because RSA is not correctly
            implemented , so a good practice for testing it is by hiring an
            security expert who knows crypto and and be doing permanent checks.

        \subsection{Symmetric encryption}

            As the communication between the users and the server have to be
            correctly cypher, the data inside the server have to be correctly
            cypher to prevent data leaks in the future , the passwords of the
            users inside the Database have to be properly hashed. 

            \\ The idea of hashing is replacing strings for random big numbers
            , this can be quite vulnerable because attackers can break the
            random function that transform characters into numbers , also
            because atackers can search for colissions, a colission occurs when
            the attacker find two diferent strings that gives the same big
            number, that number is called hash. 
            
            \\ To prevent attackers from reversing the hash algorithm, the
            company can use one of the standart hash algorithms out there,
            there are a lot of them, but one of the most commons is SHA that
            stands for Secure Hashing Algorithm.
            
            \\ To prevent attackers from bruteforcing colissions, the idea is
            to make bigger numbers, bigger numbers implies more costs at
            storing the passwords but also more security for the users, so the
            company can decide the lenght of the hashing algorithms, but they
            have to know that old algorithms that create short passwords such
            as SHA1 are totally vulnerable nowdays.

        \subsection{Using crypto for protecting employee's personal data}

            For some people all of this is not enought, so, if the company
            thinks that the above is not a sufficient condition for its users
            to be safe , then can force all the employees to encrypt their
            documents in their personal computers , there are many tools for
            doing this but i like gnu's encryption tool that is call GPG.

    \section{Basics about secure credentials}
        
        Then,
        they have to check that everyone who have access to the server have
        secure credentials. Based on how much money they have and how valuable
        is the data that they want to protect, and prioritizing the accounts
        that have more priviledges 
               
        \section{Credentials}
            They must have credentials based on 3 things:
        \begin{enumerate}
            \item {Something that they have} 
            \item {Something that they are} 
            \item {Something that they know} 
        \end{enumerate}
        \subsection{Something that they have:}
            is about something physical that cannot be replicated
            easily, such as the cellphone number or a token. 

        \subsection{Something that they are}
            is about physical characteristics that humans have such as the
            voice, the footpring, the eyes reticle ...etc . this is actually the
            most vulnerable characteristic because an attacker can artificially
            simulate many of those characteristics.

        \subsection{Something that they know}
            Something that they know is about clues that are supposed to be
            confidential for the people
            such as passwords or personal questions like the name of their pets,
            the favorite football player or the origin city.
            Actually private questions are quite vulnerable because an skilled
            attacker can guess those kind of questions by researching information
            about a targets, and passwords can be quite vulnerable too because an
            attacker can brute force it. so for making a secure password they will
            have to consider those items: 
            \\ 
        \begin{enumerate}
            \item {Password Lenght} 
            \item {Type of characters in the password} 
            \item {pronounceability of the password and relation with the owner} 
        \end{enumerate}
            \\ 
            More longer is the password, more difficult is to brute force it , if
            the password have special characters, upper and lower case letters and
            numbers it also increase the number of possible passwords for an
            attacker, the company can check how secure is a password here :
            (https://tmedweb.tulane.edu/content\_open/bfcalc.php) to check how much
            time would take to an attacker to break it using brute force it. Also, attackers uses tools like :
            \\ 
        \begin{enumerate}
            \item {Cupp(common User Password Profiler (https://github.com/Mebus/cupp))} 
            \item {Wyd (Who's Your Daddy' : (https://www.darknet.org.uk/2006/11/wyd-automated-password-profiling-tool/))} 
            \item {Crunch : https://github.com/crunchsec/crunch} 
        \end{enumerate}
        
            \\ 
            to create dictionaries for brute forcing passwords based on information that they know about the
            target. They also used dictionaries of vulnerable passwords. so a
            good practice for making good passwords is constantly
            attacking them based on information about the owner of then to
            check
            that he's not using personal information on them.

        \subsection{Multifactor Authentication , How to make secure credentials and how to take advantage of VPN services}
            By far, the most secure credentials for a system would be
            implementing those 3 things on every client, but this can be
            suffocating for users and also expensive, so based on minimum
            priviledges concept and how much priviledges have every account
            inside the system, the E-Shop administrators will have to decide
            which credential system to implement at each rung of their pyramid
            of privileges. Also. all of the clients should use trusted VPN
            services when they are on insecure networks to prevent attacks.
            
            
            \\ 
            \\ There are some tools out there that generate secure passwords ,
            to generate them, i use a module in python that is called UUID and
            it stands for Unique User ID but the company can search for better
            alternatives.

         \section{HTTPS vs HTTP}
             \subsection{HTTP}
                 Http stands for Hipertext Transfer Protocol , its a vulnerable
                 way to transfer packages, i actually dont know why yet(but i
                 promise i'll read about this later), but i
                 know that attackers can sniff the packages if they are going
                 by http protocol.

             \subsection{HTTPS and SSL certificate}
                 Https stands for Hipertext Transfer Protocol Secure , an SSL
                 certificate is a flag issued by an authority if the authority
                 knows that the website is authentic, so, if
                 the company want the users to be secure that they are in the
                 company website and there's noone outhere attacking by
                 phishing , by having https and SSL cerfiticate, the browser is
                 going to mark them with a green lock that works as a flag for
                 users to trust in the website.
                 
        

        
        

    %=======================NOTES ENDS HERE===================%
    
    % bib stuff
    \nocite{*}
    \addtocontents{toc}{\protect\vspace{\beforebibskip}}
    \addcontentsline{toc}{section}{\refname}    
    \bibliographystyle{plain}
    \bibliography{../Bibliography}
\end{document}
