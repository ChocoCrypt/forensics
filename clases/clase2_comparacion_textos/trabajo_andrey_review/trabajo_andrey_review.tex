% article example for classicthesis.sty
\documentclass[10pt,a4paper]{article} % KOMA-Script article scrartcl
\usepackage{lipsum}     %lorem ipsum text
\usepackage{titlesec}   %Section settings
\usepackage{titling}    %Title settings
\usepackage[margin=10em]{geometry}  %Adjusting margins
\usepackage{setspace}
\usepackage{listings}
\usepackage{amsmath}    %Display equations options
\usepackage{amssymb}    %More symbols
\usepackage{xcolor}     %Color settings
\usepackage{pagecolor}
\usepackage{mdframed}
\usepackage[spanish]{babel}
\usepackage[utf8]{inputenc}
\usepackage{longtable}
\usepackage{multicol}
\usepackage{graphicx}
\graphicspath{ {./Images/} }
\setlength{\columnsep}{1cm}

% ====| color de la pagina y del fondo |==== %
\pagecolor{white}
\color{black}



\begin{document}
    %========================{TITLE}====================%
    \title{\rmfamily\normalfont\spacedallcaps{ Review del trabajo de Andrey }}
    \author{\spacedlowsmallcaps{Rodrigo Castillo}}
    \date{\today} 
    
    \maketitle
     

    %=======================NOTES GOES HERE===================%
    \section{Cosas Buenas}
        es fácil de entender , explica sobre el modelo cliente-servidor,
        propone cosas ingeniosas fáciles de implementar
        y es simple , abarca la mayoría de los temas mensionados en el taller y
        cualquier persona en una empresa entendería conceptos
        básicos de seguridad informática.

    \section{Cosas que no me agradaron}
       Creo que pudo extenderse un poco mas en algunos temas , faltaron
       estrategias como el uso de VPNs , el cifrado , la  validacion de las
       paginas mediante certificado.
    
    \section{Nota}
        4.5

    







    %=======================NOTES ENDS HERE===================%
    
    % bib stuff
    \nocite{*}
    \addtocontents{toc}{\protect\vspace{\beforebibskip}}
    \addcontentsline{toc}{section}{\refname}    
    \bibliographystyle{plain}
    \bibliography{../Bibliography}
\end{document}
