% article example for classicthesis.sty
\documentclass[10pt,a4paper]{article} % KOMA-Script article scrartcl
\usepackage{import}
\usepackage{xifthen}
\usepackage{pdfpages}
\usepackage{transparent}
\newcommand{\incfig}[1]{%
    \def\svgwidth{\columnwidth}
    \import{./figures/}{#1.pdf_tex}
}
\usepackage{lipsum}     %lorem ipsum text
\usepackage{titlesec}   %Section settings
\usepackage{titling}    %Title settings
\usepackage[margin=10em]{geometry}  %Adjusting margins
\usepackage{setspace}
\usepackage{listings}
\usepackage{amsmath}    %Display equations options
\usepackage{amssymb}    %More symbols
\usepackage{xcolor}     %Color settings
\usepackage{pagecolor}
\usepackage{mdframed}
\usepackage[spanish]{babel}
\usepackage[utf8]{inputenc}
\usepackage{longtable}
\usepackage{multicol}
\usepackage{graphicx}
\graphicspath{ {./Images/} }
\setlength{\columnsep}{1cm}

% ====| color de la pagina y del fondo |==== %
\pagecolor{white}
\color{black}



\begin{document}
    %========================{TITLE}====================%
    \title{{  8th Laboratory: Modify execution  }}
    \author{{Rodrigo Castillo}}
    \date{\today}

    \maketitle


    %=======================NOTES GOES HERE===================%
    \section{Q: Read section "Debugging" and identify the difference between a
    "disassembler" and a "debugger", and between a "source-level debugger" and
    an "Assembly-level debugger".R:}
        % ====|ACA EMPIEZA EL PUNTO 1|====

        \begin{itemize}
            \item {The main difference between a disassembler an a debugger is
                that a disassambler cant run instructions, it only print them,
            a debugger run instructions step by step , but not necesary print them}

            \item {a source-level debugger is a debugger that is made for
                debugging code, so for example, visual studio code has a
            debugger, an asembly level debugger is made for undertanding
            binaries, so it run machine instructions instead of code instructions}
        \end{itemize}

        % ====|ACA TERMINA EL PUNTO 1|==== %

    \section{Debugging an application in user mode implies that the debugger
    application (WinDbg or OllyDbg) is running on the same system as the code
    being debugged. Explain how kernel debugging is performed?}

    % ====|ACA EMPIEZA EL PUNTO 2|====



    % ====|ACA TERMINA EL PUNTO 2|==== %


    \section{Q: Read the section "Pausing Execution with Breakpoints" and
    explain the differences between "Software Execution Breakpoints", "Hardware
    Execution Breakpoints" and "Conditional Breakpoints" through the following
    table.}
    % ====|ACA EMPIEZA EL PUNTO 3|====



    % ====|ACA TERMINA EL PUNTO 3|==== %

    \section{Table}

    % ====|ACA EMPIEZA EL PUNTO 4|====

        \textbf{Software execution Breakpoints:}
        \begin{enumerate}
            \item {Prupose of breakpoint:}
            \item {Number of supported breakpoints:}
            \item {How it works:}
            \item {Drawbacks:}
        \end{enumerate}

        \textbf{Hardware execution Breakpoints}
        \begin{enumerate}
            \item {Prupose of breakpoint:}
            \item {Number of supported breakpoints:}
            \item {How it works:}
            \item {Drawbacks:}
        \end{enumerate}

        \textbf{Conditional Breakpoints:}
        \begin{enumerate}
            \item {Prupose of breakpoint:}
            \item {Number of supported breakpoints:}
            \item {How it works:}
            \item {Drawbacks:}
        \end{enumerate}

    % ====|ACA TERMINA EL PUNTO 4|==== %


    \section{Q: Read the section "Exceptions" and explain the difference
    between a first and second chance exception.}

        % ====|ACA EMPIEZA EL PUNTO 5|====



        % ====|ACA TERMINA EL PUNTO 5|==== %


    \section{Q: Explain the differences between , an exception induced by a
    breakpoint, , an exception from a single-stepping and , a
    memory-access violation exception.}
    % ====|ACA EMPIEZA EL PUNTO 6|====



    % ====|ACA TERMINA EL PUNTO 6|==== %

    \section{Q: Download OllyDBG from http://www.ollydbg.de/odbg110.zip and
    PowerISO from
    http://nourinfo.com/tools/Power\%20ISO\%206.6/PowerISO6.exe. Then, crack
    PowerISO to achieve to use the application without a valid "registration
    code"}
    % ====|ACA EMPIEZA EL PUNTO 7|====



    % ====|ACA TERMINA EL PUNTO 7|==== %

    \section{R: Put here a screenshot of PowerISO displaying the message:
    "Thanks for your registration"}
    % ====|ACA EMPIEZA EL PUNTO 8|====



    % ====|ACA TERMINA EL PUNTO 8|==== %

    \section{Q: Try to open netcat using the arguments -l -p 443}
    % ====|ACA EMPIEZA EL PUNTO 9|====



    % ====|ACA TERMINA EL PUNTO 9|==== %

    \section{R: Put here the full screenshot that show "nc" running from the
    debugger and receiving messages by the port 443 sent from another console
    that has a telnet app started.}
    % ====|ACA EMPIEZA EL PUNTO 10|====



    % ====|ACA TERMINA EL PUNTO 10|==== %

    \section{Q: From the previous questions. The address of netcat's System
    Startup breakpoint is at 7C91120F. The address of netcat's Entrypoint is at
    00401160. Explain how may I control where the application being opened by
    Ollydbg is PAUSED?}
    % ====|ACA EMPIEZA EL PUNTO 11|====



    % ====|ACA TERMINA EL PUNTO 11|==== %

    \section{Q: Read the section "The OllyDbg Interface" and explain what is
    contained in each of the four windows (Disasssembler window, Registers
    window, Stack window, Memory dump window) that are opened when OllyDBG starts
    an application.}

    % ====|ACA EMPIEZA EL PUNTO 12|====



    % ====|ACA TERMINA EL PUNTO 12|==== %


    \section{Q: Read the section "(RAM) Memory Map" and explain in your own
    words the concept of "rebased" and the issues that an "absolute address"
    brings.}
    % ====|ACA EMPIEZA EL PUNTO 13|====



    % ====|ACA TERMINA EL PUNTO 13|==== %

    \section{Q: Read the section "Execute code" and explain how stepping-over
    works.}

    % ====|ACA EMPIEZA EL PUNTO 14|====



    % ====|ACA TERMINA EL PUNTO 14|==== %

    \section{Q: Read the section "Breakpoints" and explain when would it be
    useful to employ a "Conditional breakpoint" (Fig 9-8).}

    % ====|ACA EMPIEZA EL PUNTO 15|====



    % ====|ACA TERMINA EL PUNTO 15|==== %

    \section{Q: Read section "Loading DLLs" and try uploading a DLL malware
    file which we know has an export function.}
    % ====|ACA EMPIEZA EL PUNTO 16|====



    % ====|ACA TERMINA EL PUNTO 16|==== %



    % ====| ACA EMPIEZA LA PARTE PRACTICA DIOOOS QUE TALLER TAN GONORREA |==== %



    \section{Q: What is the purpose of the argument -in?}
    % ====|ACA EMPIEZA EL PUNTO 17|====



    % ====|ACA TERMINA EL PUNTO 17|==== %

    \section{Q: Which situations could provoke that the malware delete itself?}
    % ====|ACA EMPIEZA EL PUNTO 18|====



    % ====|ACA TERMINA EL PUNTO 18|==== %

    \section{Q: Achieve that that the malware gets executed even if not having
    the right password}
    % ====|ACA EMPIEZA EL PUNTO 19|====



    % ====|ACA TERMINA EL PUNTO 19|==== %


    % ====| ACA ACABA EL TRABAJO |==== %
    %=======================NOTES ENDS HERE===================%

    % bib stuff
    \nocite{*}
    \addtocontents{toc}{{}}
    \addcontentsline{toc}{section}{\refname}
    \bibliographystyle{plain}
    \bibliography{../Bibliography}
\end{document}
